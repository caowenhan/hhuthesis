%% This is file 'abstract.tex'
%% It is included by hhuthesis-example.tex for hhuthesis.
%%
%% Copyright(C) 2020, Wenhan Cao
%% College of Water Conservancy and Hydropower Engineering, Hohai University.
%%
%% Version:v1.0.0
%% Last update: July 19th, 2020.
%%
%% Home Page of the Project: https://github.com/caowenhan/thesis
%%
%% This file may be distributed and / or modified under the conditions of the
%% LaTeX Project Public License, either version 1.3c of this license or (at your
%% option) any later version. The latest version of this license is in:
%%
%% http://www.latex-project.org/lppl.txt
%%
%% and version 1.3c or later is part of all distributions of LaTeX version
%% 2008/05/04 or later.
%%

\begin{abstract}
本文综合应用数学和力学理论、坝工知识和大坝安全综合评价理论等,采用数值仿真、室内与现场试验、原型监测资料分析等手段,系统开展了寒冷地区渗水病害影响下混凝土坝服役性态多尺度分析方法研究,主要研究内容如下:
\begin{enumerate}
	\item[(1)] 引入HYMOSTRUC 3D水化模型,研究了坝体混凝土材料微观尺度结构特性;采用静水压力和结晶压力复合作用模型,模拟了坝体混凝土材料冰冻破坏特性,据此分析了冻融影响下坝体混凝土材料微观力学性能演化规律;结合坝体混凝土材料渗透溶蚀退化机制分析,提出了水泥基材料析钙溶蚀模型,并与随机溶蚀算法有机融合,探究了坝体混凝土微观溶蚀力学性能演化特性。
	\item[(2)] 采用多尺度摄动技术,将坝体混凝土材料多尺度力学问题转化为多尺度均匀化联合方程求解问题,由此提出了坝体混凝土多尺度力学递进分析模型;基于坝体混凝土材料在冻融与溶蚀影响下微观力学性能演化特性分析,并以连续损伤力学理论为支撑,采用多尺度能量积分方法,提出了坝体混凝土材料多尺度损伤表征方法,实现了寒冷地区渗水病害影响下坝体混凝土材料力学性能多尺度传递分析。
\end{enumerate}


\keywords{寒冷地区;混凝土坝;服役性态;渗水病害;多尺度分析}
\end{abstract}

\begin{enabstract}
The integrated application of methmatics and mechanics theory, dam engineering theory and dam comprehensive safety evaluation theory were introduced to systematically evaluate and research the concrete dam performance evolution under the influence of water-seepage diseases by adopting numerical simulation, indoor and field test and field data analysis method with multi-scales method. The main research contents were as follows:

\begin{enumerate}
\item[(1)] The micro-scale structural characteristics of concrete material were investigated by introducing the hydration model HYMOSTRUC 3D. Based on the frozen damage mechanism of the concrete material, the composite model of hydrostatic pressure and crystallization pressure was adopted to research the micro-mechanical concrete mechanical evolution properties; based on the corrosion degradation mechanism of the concrete under the corrosive environment, the calcium precipitation dissolution model of the cement-based material was established. By integrating with stochastic solution algorithm, the micro mechanical degradation characteristics of the concrete material was studied.
\item[(2)] By adopting the multi-scale perturbation technique, the multiscale mechanics problem of hydraulic concrete was transformed into solution problem of multiscale homogenization combination equation and the multiscale mechanical progressive model was thus established; based on the micro mechanical characteristics of the concrete material under the effects of freezing and thawing and corrosion, the multi-scale damage theory of concrete material was established by synthesizing continuum damage mechanics theory and multi-scale energy integral method, which could successfully realize the multi-scale mechanical properties transmission of concrete material in cold region under the effect of typical water-seepage diseases.
\end{enumerate}  
 
\enkeywords{cold region; concrete dam; service ability; water-seepage diseases; multiscale analysis}

\end{enabstract}
